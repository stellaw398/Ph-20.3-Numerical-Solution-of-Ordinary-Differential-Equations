\documentclass{article}
\usepackage{pythontex}
\usepackage{graphicx}
\usepackage{verbatim}
\usepackage[usenames,dvipsnames,svgnames,table]{xcolor}
\newcommand{\cdw}[1]
           {{\color{blue} [CDW: comment] #1}}
\newcommand{\correct}[1]
           {{\color{red} [CDW: needs correction] #1}}
\title{Ph 20.3 - Numerical Solution of Ordinary Differential Equations
\\\correct{}}
\author{Stella Wang}
\date{ 7 February 2018}
\begin{document}
\maketitle
\correct{You need to include the Python source and the git log in this pdf}
\verbatiminput{log.log}
this is a git log 
\section{Part 1}

Below is a plot of x and v as a function of time for both the implicit and explicit Euler method. From the plot it is clear that the error grows rapidly with each cycle for both methods. The explicit oscillations increase in amplitude while the implicit oscillations decrease in amplitude. For both methods an $h = 0.1$ was used. (Figure 1)

\begin{pylabcode}[firstsession]
import numpy as np
import matplotlib.patches as mpatches
N = 200
t=0
x_olde=5
v_olde=.5
x_oldi = x_olde
v_oldi = v_olde
h= .1
det = 1/(1 + (h)**2)
for i in range(N+1): 
    t_new = t + h*i
    X= 5*np.cos(t_new)
    V= -5*np.sin(t_new)
    x_newe = x_olde +(h*v_olde)
    x_newi = det*(x_oldi +(h*v_oldi))
    v_newe = v_olde -(h*x_olde)
    v_newi = det*(v_oldi-(h*x_oldi))
    Ee= (x_newe)**2 + (v_newe)**2
    Ei=(det*x_newi)**2 + (det*v_newi)**2
    E = X**2 + V**2
    figure(0)
    plot(t_new,x_newe,'bo');plot(t_new, v_newe,'ro')#explicit
    plot(t_new,x_newi,'ko');plot(t_new, v_newi, 'yo')#implicit
    figure(1)
    plot(t_new,X-x_newe,'bo' );plot(t_new,V-v_newe,'ro') #explicit error
    plot(t_new,X-x_newi,'ko' );plot(t_new,V-v_newi,'yo')#implicit error
    figure(2)
    plot(t_new,Ee, 'g*'); plot(t_new,Ei,'r*'); plot(t_new, E, 'ko')
    figure(3)
    ##phase space
    plot(x_newe, v_newe,'r<'); plot(x_newi, v_newi, 'y>' ); plot(X,V,'bo')
    x_olde = x_newe
    v_olde = v_newe
    x_oldi = x_newi
    v_oldi = v_newi
x_old=5
v_old=.5  
##symplectic Euler
for i in range(0,N+1):
    t_new = t + h*i
    x_new = x_old + h*v_old
    v_new = v_old - h*(x_new)
    X= 5*np.cos(t_new)
    V= -5*np.sin(t_new)
    Es = (x_new)**2 + (v_new)**2
    E = X**2 + V**2
    figure(3)
    plot(x_new,v_new, 'g*')
    figure(4)
    plot(t_new,Es, 'bo')
    x_old = x_new
    v_old = v_new
figure(0)
xlabel('time')
blue_patch = mpatches.Patch(color='blue', label='x explicit euler')
red_patch = mpatches.Patch(color='red', label='v explicit euler')
black_patch = mpatches.Patch(color='black', label='x implicit euler')
yellow_patch = mpatches.Patch(color='yellow', label='v implicit euler')
legend(handles=[blue_patch, red_patch, black_patch, yellow_patch])
savefig('HarmonicMotion.pdf')
figure(1)
xlabel('time')
blue_patch = mpatches.Patch(color='blue', label='x explicit euler error')
red_patch = mpatches.Patch(color='red', label='v explicit euler error')
black_patch = mpatches.Patch(color='black', label='x implicit euler error')
yellow_patch = mpatches.Patch(color='yellow', label='v implicit euler error')
legend(handles=[blue_patch, red_patch, black_patch, yellow_patch])
savefig('Globalerror.pdf')
figure(2)
xlabel('time')
ylabel('Energy')
green_patch = mpatches.Patch(color='green', label='Explicit Energy')
red_patch = mpatches.Patch(color='red', label='Implicit Energy')
black_patch = mpatches.Patch(color='black', label= 'Analytic Energy')
legend(handles= [green_patch, red_patch, black_patch])
savefig('Energy.pdf')
figure(3)
xlabel('X')
ylabel('V')
red_patch = mpatches.Patch(color='red', label=' Explicit Euler')
yellow_patch = mpatches.Patch(color='yellow', label='Implicit Euler')
blue_patch = mpatches.Patch(color='blue', label='Analytic')
green_patch = mpatches.Patch(color='green', label='Symplectic Euler')
legend(handles=[red_patch, yellow_patch, blue_patch, green_patch])
savefig('PhaseSpace.pdf')
figure(4)
xlabel('time')
ylabel('Energy')
blue_patch = mpatches.Patch(color='blue', label='Symplectic Energy')
legend(handles= [blue_patch])
savefig('EnergySym.pdf')

\end{pylabcode}

\begin{figure}[h]
\includegraphics[width= \linewidth]{HarmonicMotion}
\caption{Simple Harmonic Motion using Explicit and Implicit Euler method}
\end{figure}

The analytic solution for the Simple Harmonic Oscillator is of the generalized form $ x(t) = A \cos( \omega t + \phi)$ and $ v(t) = -A \omega \sin( \omega t + \phi)$. For our system with initial conditions $x_o = 5$ and $v_o = 0.5$ it is found that  $A=5$, $\omega = 0$ and $ \phi = 0$. A plot of the global error for both the implicit and explicit Euler method is shown below. 
It can be seen that the error for implicit Euler method is lower than the error for explicit Euler method, but both grow with each cycle.(Figure 2)
\begin{figure}[h]
\includegraphics[width= \linewidth]{Globalerror}
\caption{Global Error}
\end{figure}
The total energy of the system $E = x^2 + v^2$ should be constant. Plotted below is the energy as a function of time for explicit and implicit euler and the analytic solution. The analytic solution is constant as we would expect, while the implicit euler method decreases (similar to the overall harmonic motion) and explicit euler increases respectively. (Figure 3)
\begin{figure}[h]
\includegraphics[width= \linewidth]{Energy}
\caption{Total Energy}
\end{figure}

The truncation error for explicit error is shown in figure 4. The error is plotted versus h for $h_o, h_o/2, h_o/4, h_o/8$ and $ h_o/16$ , $h_o=0.1$.The plot is log-log and is nearly linear, showing that for reasonably small $h$, the error is proportional. (Figure 4)




\begin{pylabcode}[secondsession]
#
import numpy as np
h=0.1
for j in range (5):
	h = h/(2**j)
	N = 200
	t=0
	x_olde=5
	v_olde=.5
	truncation=[]
	for i in range(0,N+1): 
		t_new = t + h*i
		X= 5*np.cos(t_new)
		V= -5*np.sin(t_new)
		x_newe = x_olde +(h*v_olde)
		v_newe = v_olde -(h*x_olde)
		truncation.append(abs(X-x_newe))
       		x_olde = x_newe
       		v_olde = v_newe
	maxerror=max(truncation)
	plot(h, maxerror, 'bo')   
semilogy()
semilogx()
xlabel('h')
ylabel('truncation error')
savefig('truncationerror.pdf') 
\end{pylabcode}
\begin{figure}
\includegraphics[width= \linewidth]{truncationerror}[h]
\caption{Truncation Error versus h}
\end{figure}
\section{Part 2}

The phase space diagram for all the methods is shown below. The analytical solution is a circle, the explicit euler spirals outward, the implicit euler spirals inward and the Symplectic euler method is nearly circular (it does not perfectly line up) (Figure 5)
The total energy for Symplectic Euler oscillates around the constant total energy value. This makes sense, because the phase space shows that symplectic euler is elliptic. An ellipse is a circle with a phase, so naturally the energy should oscillate. (Figure 6)

\begin{figure}
\includegraphics[width= \linewidth]{PhaseSpace}
\caption{Phase Space}
\end{figure}
\begin{figure}
\includegraphics[width= \linewidth]{EnergySym}
\caption{Total Energy for Symplectic Euler}
\end{figure}
\end{document}
